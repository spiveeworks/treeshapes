\documentclass{report}
\usepackage{dsfont}
\begin{document}

\section{Meeting}

inspiration project dealt with 3 different matrix representations of trees,
spectra are invariant under relabeling in all of these representations, which
are simply conjugation by some permutation matrix (similar by definition)
however all of these representations fail to be fair representations: there are
trees with different shape and the same spectrum A motivation for
distinguishing these trees would be measuring their distance, perhaps as the
norm of the difference between the spectra

Another inspiration shows that trees can be represented as algebras, as the
span of the matrices generated by $Rel(a, b)$ iff $Sup(a, b) = x$ for fixed
node x in tree, each x gives a different matrix, their span is a commutative
algebra

then being a commutative algebra, each basis is simultaneously diagonalizable,
in other words, a linear combination of the basis matrices has a spectrum
corresponding to the same linear combination of spectra of the basis matrices.
This makes the spectrum an appealing representation, since as before
relabelling the tree shouldnt change the spectrum, but merely permute the
eigenvectors.

One consideration with spectral representations, is how to order the
eigenvalues, and their corresponding eigenvectors.  When saying that two
matrices have the same eigenvalues, what one means is that there is an ordering
of one matrices eigenvalues that correspond identically with the eigenvalues of
the other matrix. This will account for the multiplicity of each eigenvalue.
This suggests 3 ways of canonicalizing the spectra, one is to check the
necessary condition that each eigenvalue in the first has at least one
corresponding eigenvalue in the second, by representing the spectra as a set
of eigenvalues. This shall be immediately discarded, since it could so easily
compromise the injectivity of the representation, since
$\{1, 1, 2\} = \{1, 2, 2\}$

The other two ways are to sort the list (choose a canonical rearrangement) and
equivalently to use a multiset.

Sorting the list will definitely work for the single matrix representations
above, but for the algebra representation, this would compromise the
motivating linearity property that emerges from the simultaneous
diagonalizability of the algebras.
In particular $<1, 1> - <0, 1> = <1, 0>$ which isn't sorted.
Sorting the result will restore the equality, but at this point the set of
spectra is looking a lot less convenient, though it could still be of interest.

Without sorting, the image of the algebra forms a vector space, whose dimension
could range from 1 to n-1 where n is the number of leaves on the trees in
question.
Since the matrices are simultaneously diagonalizable, we can pick a single set
of eigenvectors in advance, and use this to compute all the spectra, resulting
in a single vector space, however different permutations of these eigenvectors
will be valid and will permute the coordinates in $\mathds{C}^n$.

It is unclear how to apply this to get a distance formula..

so the question of identifying similar spectra becomes a question of similar
vector spaces of spectra\ldots from this perspective it is clear that sorting
no longer works.

In any case then the important stuff:
\begin{itemize}
	\item subspace has at most n-1 dimensions
	\item subspace could have less if there is degeneracy
	\item in particular if two subtrees have the same shape then their roots
		will have the same corresponding matrices.
	\item further all algebras of a given n will have one consistent
		eigenvalue of 0,
	\item this means that if there is no degeneracy the image will always be
		$\mathds{C}^{n-1}$
	\item this means the representation is definitely not injective, however
		the dimension will still be a measure of degeneracy
	\item also degenerate trees may have distinct images?
\end{itemize}

\end{document}
